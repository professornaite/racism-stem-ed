

A global network of scholars has identified how racism deeply impacts the well-being of communities, and the families and individuals that make up those communities. 
These researchers increasingly cross disciplinary boundaries to make sense of how racism is both studied and understood in different contexts and across various ecosystems, showing racism's widespread impact and a need to make sense of the study of racism. 
The work of interdisciplinary scholars identifies how racism holds consequences that are unique to each of the historical, social, and cultural contexts in which racists attitudes and actions manifest, while also being notable across social and political boundaries.
These studies extend our general understanding of racism while also providing new insights into increasingly focused contexts.

The evolving notions of racism in contemporary sociology contribute to an expanding body of interdisciplinary research on racism in science, technology, engineering, and mathematics (STEM).
The methods used by this interdisciplinary network of scholars to develop insights about the evolving conceptualizations of racism, however, tend to be discipline specific.
As social scientists increasingly engage in interdisciplinary scholarship, the traditional methods used to identify themes within an area of study may fail to engage diverse voices and scholarship.
This disciplinary specificity has led to fragmented understandings of racism and can possibly cause scholars to overlook important cross disciplinary insights.
As computational methods continue to advance in the social sciences, there is a need for  integrated conceptual approaches that help frame the diversity of research across STEM fields while also acknowledging the theoretical traditions which speak to the foundations of the study of racism.


Contemporary sociology and education research on racism have contributed significantly to our framing and understanding of racial inequities in STEM, which is evident by the expansion of systematic studies on racism [@shiao2021]; however, the reach of these rapidly expanding bodies of scholarship is not well-documented.
One potential cause of this gap in knowledge is due to disciplinary differences in methodological approaches, theoretical traditions, and analytic frameworks.
Understanding the various themes and topics across various scholarly outlets will not only help synthesize diverse perspectives it will provide researchers with the opportunity to capture the complex, multifaceted nature of racism across various STEM contexts.


In this article, we examine the differences across sources and strategies to model the notions derived from metadata in journal article content. 
Beyond the traditional uses of systematic literature reviews to track the frequency of publications over time and examine citation patterns and influence, we integrate text mining techniques to identify common themes and concepts, or notions. 
These notions lend to our understanding of the ways that computation can be used to understand the various concepts regarded across seemingly disparate networks and citation communities. 
By analyzing seemingly disparate networks and citation communities, our approach can uncover connections and commonalities in how racism is understood and addressed across different STEM fields, potentially highlighting the contributions of education research that is often undervalued in STEM journals. 
This computational analysis of notions can reveal how concepts of racism in STEM have evolved over time and across disciplines, creating a more holistic understanding that can facilitate better interdisciplinary communication and collaboration on addressing racism in STEM. 

When studies of racism are discipline specific, scholars across these different disciplines approach both the study of racism and the systematic review of literature on racism in distinct ways.

The nuance attributed to the various injustices and their intersections has contributed to an emerging and interdisciplinary set of scientific studies.

Despite this emerging literature sharing 'racism' as one common topic of inquiry, the body of work is increasingly unwieldy and additional perspectives are needed to help incorporate intersecting meanings but different notions across the disciplines, often noted by the use of keywords and context.

It is well known , and often within a specific area of study multiple methods and tools may be offered to help advance an area of inquiry.
This dual reality, of increasingly interdisciplinary perspectives alongside advancing rather focused lines of inquiry, can be attributed to the historical traditions of expansion and the particular needs of a discipline. 
The disciplinary traditions tend to be prioritized in the development of new knowledge, supporting important insights within a specific academic framework. 
However, this approach may inadvertently limit the full acknowledgment of the expansive and diverse perspectives that can be developed across disciplines, potentially overlooking valuable interdisciplinary perspectives. 
The use of different computational methods and databases further skews the analysis and results of analytic methods applied to a corpus of data [@Antons_2021]. 

# Theoretical Framework

The approaches to systematic analyses vary by discipline for a few key reasons. 
First, the different historical contexts and discipline-specific content speak to the varied theoretical traditions, methodological approaches, and research paradigms.
For instance, STEM fields traditionally rely on quantitative methods, while education research often incorporates qualitative and mixed-methods approaches to capture nuanced experiences of marginalized groups.
The broad structure of a discipline and its specific sub-cultures also result in a secondary set of distinct and evolving analytic frameworks, as well as concepts, that inform the practices within a specific intersection or sub-area of study.
This is evident in the emerging use of critical quantitative approaches, like QuantCrit [@Garcia2018], in STEM, which challenges traditional paradigms in service of new interdisciplinary perspectives.
Together, these interactions produce a complex system of information that has become a problem in effectively capturing cross-disciplinary insights and diverse voices.

Studies of racism in contemporary sociology offer a range of frameworks that describe a set of specific conceptualizations of racism within an historical line of inquiry.
This inquiry rests on more than a general understanding of sociology, but on the agreed upon definitions that help to advance specific lines of research which rests on an additional set of common goals and assumptions.
This structure is not unique to sociology but, in reference to racism, this construction describes the sociology of race and ethnicity, where the concepts and lexical structures employed by authors often emerge from large-scale citation analyses [@shiao2021]. 

Meta-analytic studies have noticed that 'high-impact' studies contribute to various ideological forces in STEM [@li2022systematic; @Takeuchi_2020]. In large-scale citation studies, dynamic representations of research literature supports how scholars understand and approach cross-disciplinary research, and the various methods render insight to the intellectual networks, notions, and complex themes within a specified body of work and the use of key terms or phrases [@clair2015sociology; @shiao2021].

New technological advancements offer researchers powerful means to analyze large-scale data for key terms and phrases, where they can uncover  patterns and generate novel perspectives on complex issues like racism. 

## Conceptual framework

Some text goes here.

Table: **Table 1**. Conceptualizations and meanings of racism

| **Bonilla-Silva** (1997) | **Shiao and Woody** (2021)               | 
|:-------|:------------|
| A: Psychological phenomena   | A: Individual attitudes                |
| B: Cultural processes        | B: Cultural schema                     |
| C: Social structure           | C$_1$: Structure: Pre-existing conditions |
|                   | C$_2$: Structure: Create or maintain      |
|                   | C$_3$: Structure: Cultural mobilization                |
|                   | C$_4$: Structure: Racial dominance                 |



## Researching racism

The field of sociology offers the core set of theoretical perspectives to examine racism. Many sociologist continue to develop these frameworks further as they provide valuable insights into the complex nature of race and its impact around the globe and on different societies. There are various ideas attributed to racism across different societies and how studies around racism are structured is largely a factor of knowledge production [@acharya2022race; @bonilla2021racism; @solomos2022race]. The example of functionalism, as one classic sociological perspective that views society as a complex system whose parts work together to promote an output, explores how racism can contribute to social solidarity among in-group members, while conflict theory focuses on how racial inequalities perpetuate the power disparities between groups [@calnitsky2023class; @banton2018concept; @bonilla2001anything]. Contemporary sociological research increasingly emphasizes the study of racism as individual- and group-level processes and structures that reproduce racial inequality in subtle and diffuse ways [@bonilla2021new]. This approach allows for a nuanced understanding of how racism operates in a purportedly "post-racial" society but the meanings signaled across different areas of study are not the same.

As research on racism in STEM fields expands, it is crucial to adopt interdisciplinary approaches that can capture the multifaceted nature of racial discrimination. This may involve integrating perspectives from sociology, economics, and other social sciences to develop more comprehensive frameworks for understanding and addressing racism in these fields. Systematic studies and the analysis of corpora of historical research literature and related data offer research scientists ways to understand the study of various phenomena. However, the diverse topics encompassed within what is conceived as 'STEM', while useful for framing a set of highly related disciplines, may fail to provide a common framework for the analysis of scientific literature. As research on racism expands across all fields of study, the framing used around various concepts will offer important insights not only for future studies but also to inform policies in an increasingly divisive society. The insights provided by these studies will be crucial for developing a more inclusive and equitable approach to scientific inquiry and policy-making.


For example, the concept of institutional discrimination has gained prominence, highlighting how past discriminatory practices can have lasting consequences without requiring contemporary actors to actively discriminate.

## Study background

I will then cite @abend2008meaning.

# Methods and Data

I will cite @thunder2016 in this article.

# Conceptual Framework

Other researchers in STEM education have discussed the value of qualitative metasynthesis [@thunder2016].

Thunder & Berry [-@thunder2016] discuss this in their paper.

# Data and Methods

Some text will go here.

```{r, fig-align: center}
knitr::kable(head(mtcars[, 1:4]), "pipe",
             caption = "A sample table")
```

## Analytic framework

# References

<div id="refs"></div>

\newpage

# Appendix

![Analytic framework for the study](test.png){width=50%}


![Three field plot by country](three-field-author-country.png){width=50%}






The persistent issue of racism continues to plague societies worldwide, despite extensive research and policy efforts aimed at combating discrimination. In the United States, critical scholars have identified deep-rooted foundations of racism throughout the nation's history, highlighting its systemic functions across various institutions. One system profoundly impacted by racism is education, particularly concerning equal access and treatment within educational institutions and classroom settings. Within the broader context of educational inequity, racism in Science, Technology, Engineering, and Mathematics (STEM) fields has become a central focus for critical scholars. This heightened attention is due to society's increasing dependence on technology and computation, making equitable access to STEM education and careers crucial for social and economic progress.

The pervasive nature of racism in STEM education manifests in multiple ways, from structural barriers to implicit biases. A groundbreaking study published in PNAS Nexus revealed stark quantitative evidence of structural racism in academic STEM programs across the United States. The research found that even when controlling for equal high school preparation, white males were more likely to earn STEM-related degrees compared to their peers from underrepresented groups. Furthermore, the culture within STEM fields has been identified as a racial hierarchy, with white and some Asian individuals at the top and Black and Indigenous individuals at the bottom. This normalized system often goes unquestioned and unexamined, perpetuating inequities and discouraging diverse participation in STEM.

Given the critical importance of STEM fields in shaping our future and the persistent racial disparities within them, there is an urgent need for more comprehensive research. This research should aim to understand how racism in STEM has functioned historically and its potential impact on the future of these fields and society at large. By examining these issues through frameworks such as Critical Race Theory, scholars can work towards dismantling systemic barriers and creating more inclusive and equitable STEM environments.

*What is the state of the intellectual structure and emerging notions in research on racism in science, technology, engineering, and mathematics?*


# Study background

In examining notions of racism in STEM, it is crucial to ground our understanding in robust theoretical frameworks. Eduardo Bonilla-Silva's conceptualization of racism provides a powerful lens through which to analyze the structural components of racial inequality in STEM fields and education. Bonilla-Silva argues that racism should be understood not merely as individual prejudice or discriminatory actions, but as a structural phenomenon deeply ingrained in society's fabric. He introduces the "racialized social system approach," which posits that racial groups are hierarchically ordered in society, leading to social relations and practices that reinforce these positions. This approach is particularly relevant to STEM fields, where racial disparities in participation and achievement persist despite efforts to increase diversity and inclusion.

Central to Bonilla-Silva's work is the concept of color-blind ideology, which he identifies as a key mechanism for maintaining racial inequality in contemporary society. This ideology operates through four main frames: abstract liberalism, naturalization, cultural racism, and minimization of racism. In the context of STEM, color-blind ideology may manifest in the belief that science and technology are inherently objective and meritocratic, obscuring the ways in which systemic racism can influence access, opportunity, and recognition within these fields. Bonilla-Silva's emphasis on the subtle and sophisticated practices that perpetuate racial inequality is particularly pertinent to STEM, where overt discrimination has largely been replaced by more nuanced forms of exclusion and bias.

Bonilla-Silva's work also highlights the importance of examining institutional racism in academia, including the critique of "white logic" and "white methods" in research. This perspective is crucial for our study, as it encourages a critical examination of how racism may be embedded in the very methodologies and epistemologies that underpin STEM research and education. Furthermore, Bonilla-Silva's concept of "racial grammar" – the way racial ideology is embedded in language and discourse – provides a valuable tool for analyzing how notions of racism are articulated and contextualized in STEM and STEM education journals. By applying these theoretical insights to our bibliometric and qualitative metasynthesis methods, we can develop a more nuanced understanding of how racism operates within STEM fields, potentially uncovering systemic barriers and informing more effective strategies for promoting equity and inclusion in these crucial areas of study and practice.

We report findings from a conceptual replication of Shiao and Woody's (2021) study, *The Meaning of 'Racism'*, which examines three constructs that frame sociologists' use of the term ``racism.'' We use a quantitative historical lens to examine the framework's mapping on to racism-related research across the various science, technology, engineering, and mathematics (STEM) sub-disciplines. We map the intellectual structure of racism-focused research in STEM and STEM education journal articles and model similarities across database samples. Results indicate that research on racism in the STEM disciplines varies by ego-centric network but not necessarily by associated content. Despite the charge to examine more systemic forms of racism, the samples are primarily inclusive of research on two of the three constructs: individual attitudes and cultural schema. Three distinct citation-based communities were identified across samples. There is a change in the variance of the network citation metrics when themes are varied. We discuss the need for more empirical studies on the structural conceptions of racism and present a model of a subgroup of highly-networked authors.

Studies of racism in contemporary sociology offer a host of frameworks to describe the potential *meanings* of racism in the sociology of race and ethnicity. As research on racism in science, technology, engineering, and mathematics (STEM) continues to gain traction, the dynamic developments and interdisciplinary perspectives leveraged by scholars will require the use of dynamic frameworks that incorporate diverse voices. In large scale studies of citation networks, dynamic representations of research literature help map the increase of cross-disciplinary research and intersecting themes (Cite). These systematic studies and the analysis of corpora of bibliometric data offer research scientists ways to understand the study of the various phenomena. However, the diverse topics across what is conceived as `STEM', while useful for framing a set of highly related disciplines, fails to provide a common framework for the analysis of scientific literature. As research on racism expands, the framing used around various concepts in relation to STEM will offer important insights. These insights can be understood through an analysis of the various notations scholars have situated around the study of racism.

Some examples and widely cited studies that focus on racism and STEM include research on the experiences and perceptions of groups racialized as non-white [@basile2019they] [@carlone2007understanding; @chen2018models] [@chavez2019healing; @gaston2013impact] [@harrell2011multiple; @gray2012intersecting] @mcgee2021black;@mcgee2017burden; studies on various cultural schemes, conceptions of race and racialization, and intersecting identities [@aldana2019youth] 2019; \cite{dancy2020undergraduates}, 2020; \cite{Leyva2022}, 2022; \cite{Leyva2023}, 2023; \cite{mcgee2016devalued}, 2016; \cite{mcgee2017troubled}, 2017; \cite{nasir2017stem}, 2017; \cite{tate2020whiteliness}, 2020; \cite{wen2020effects}, 2020); and studies that analyze various structural dimensions, often examining interlocking systems and power relations [@bullock2017] [@mcgee2020interrogating] [@morton2022critical] [@vakil2019racial] [@vossoughi2018toward]. Importantly, the discussions across this extensive body of work often reside in multiple thematic areas; and many studies on racism and STEM provide insight into the ideologies, habits and traditions, social and cultural processes, and core structural components of racialization. These studies also highlight the dynamic and context-dependent nature of racism across social contexts, physical settings, and complex ecosystems [@Higgins2018]. Yet, questions remain about how scholars situate ``racism" across different disciplines of study [@bonilla2021 @shiao2021]. 

Additionally, as research on racism in STEM extends globally, questions about geographical variation and citation network patterns arise. These patterns are further influenced by discipline-specific cultures and the array of social politics around citation practices in academic research[@dion2018gendered @mott2017]. In light of these considerations, a set of key questions emerged for our research team: Through what lenses might scholars navigate the expanding research literature and intersecting areas of study that cross disciplinary boundaries? How can research scholars address the complex dynamics of increasingly niche disciplinary terminologies in light of new computational tools? And how can individual researchers and teams make sense of intersections in terminology that crosses several academic disciplines? Through these questions, our analysis of racism in STEM has implications for both theory and practice. Moreover, this analysis contributes to our knowledge about the various features that connect studies of racism in research regarding STEM.

For the purposes of this paper, we view notions as a web of concepts that relate to distinct and overlapping sets of ideas or themes. Notions around a broad topic, like racism, generate a web of concepts and terms. This web produces an interconnected and complex structure of words which imply that notions are not isolated ideas but are interrelated and interdependent. A thematic network map, as opposed to a word cloud, would be one example. In this paper, we model different approaches to generate latent constructs (or themes) as they relate to various measures of association between the metadata of STEM and STEM education journal articles which focus, in some way, on racism-related matters. We discuss the various selection and inclusion criteria, and present models on different associations. Findings reveal a set of ``steady-state" results across all models which structure the set of core notions of racism.

[@berry2013promise] talk about this in their paper.

Berry & Thunder [-@berry2013promise] discuss this in their paper.



## Defining race

The interdisciplinary framing of STEM and the term's use across both geographical and theoretical boundaries establishes a need to explore how discussions of racism have been codified in ``STEM" and in the broader collection of critixal\footnote{An `x' is used here primarily as a literary tool -- and as a variable indicator -- to note that various meanings and conceptions of ``critical" can be identified in scholarly research. It is beyond the scope of this study to define the term `critixal' but we point readers to \cite{apple2010theory} (2010) for a sample of possible discourses around the general treatment of the term critical in education research.} studies on science, technology, engineering, and mathematics. Here, we notice that studies of racism have also been situated in broader engagements with the sciences, technology, medicine, engineering, mathematics, psychology, and the many STEM related disciplines. We use the term \textit{notion} to identify the context-rich evidence identifiable in the syntactical structures and keywords used by researchers across disciplines, and their diverse representations of racism (\cite{bonilla2021}, 2021). These representations also relate to conceptual and methodological inquiries which serve to inform how different meanings may be constructed. These questions, given the increasingly computationally-dependent nature in studies of bibliometric data and citations, systematics reviews, and meta-analytic studies, calls for further engagement with  different analytic priorities across disciplinary areas (\cite{shiao2021}, 2021).

Research by \cite{bonilla2021} (2021) identified the various ways that scholars structure their discussions of racism, a notable difference between, for example, individual attitudes and broader systems. As studies of racism expand in a field, these scholars have identified how the diverse interpretations of ``racism" call for examining the various \textit{meanings} of ``racism" in a body of research literature (\cite{bonilla2021}, 2021; \cite{shiao2021}, 2021). We examine notions of racism in STEM using a context-based bibliometric approach.

As scholars better understand the meanings prescribed to racism and the presence of moving syntactical signifiers, there is also a need to understand the history and foundations of popular policies and the use of specific terms, like STEM (\cite{gil2020stem}, 2020). We examine notions of racism as framed in a body of research on ``STEM," and in mathematics, science and technology studies to make sense of how racism is framed and contextualized. For instance, \cite{benjamin2016catching} (2016) uses the term `STS', instead of STEM, to refer to the specific area of science and technology studies (STS) (\cite{york2018}, 2018). However, despite focusing on two of the integral dimensions of STEM  -- science and technology -- many traditional approaches in bibliometrics and citation analysis may overlook this contribution due to differences in terminology, which is a common challenge in cross-disciplinary research (CDR, \cite{donovan2015your}, 2015; \cite{takeuchi2020transdisciplinarity}, 2020). 

## Framing racism

Notions, then, are a function of the lexical patterns identified in a set of keywords-in-context. With increases in latent semantic analysis (LSA) and other methods, the development of broad themes in a body of work provides content for further inquiry, and not full results. When themes and primary concepts are situated as the meanings, care must be taken with the broader social and political context of the disciplinary boundaries. For example, Shiao \& Woody respond to the development of the journal \textit{Name} and a call by Bonilla-Silva (1997). In U.S. mathematics education, the social turn (Lerman, 2000), sociopolitical turn (Gutierrez, 2007), and spatial turn (Cite, XXXX) prompted the inclusion of different methods for the analysis of data. We observe similar patterns across other fields of study. As a result, notions provide a sampling of the ideas that exist across samples with the understanding that different disciplines both uptake and examine developments in the disciplinary area largely from disciplinary perspectives. As interdisciplinary research expands, additional models and frameworks will be required.

Based on the search results, sociologists define racism in several different ways across various communities of scholarship:

1. Structural/Institutional Approach:
Some sociologists, like Bonilla-Silva, define racism as a structural phenomenon ingrained in society's institutions and systems, rather than just individual prejudice. This view sees racism as "a structure, a network of relations at social, political, economic, and ideological levels that shapes the life chances of the various races" [1].

2. Cultural Approach: 
Other scholars focus on racism as cultural messages, ideologies, and practices that produce and normalize racial inequities. For example, Ibram X. Kendi defines racism as "a marriage of racist policies and racist ideas that produces and normalizes racial inequities" [1].

3. Individual Attitudes/Psychology Approach:
Some earlier definitions focused more on racism as individual-level prejudice, bias, or hostility toward other racial groups. This view sees racism primarily as an individual problem of overt hostility [1].

4. Group Conflict Approach:
Some sociologists view racism as rooted in group-level competition over resources and power between racial groups [1].

5. Intersectional Approach: 
Scholars like Patricia Hill Collins examine how racism intersects with other systems of oppression like sexism and classism [1].

6. "New Racism" Approaches:
More recent theories look at subtle, covert forms of racism that persist in an era of declining overt racist attitudes, including concepts like colorblind racism, laissez-faire racism, etc. [1][2]

The analysis by Shiao and Woody (2021) identified six distinct communities of scholars using different meanings and citations related to racism, spanning structural, cultural, and attitudinal conceptions [4]. They found that even though Bonilla-Silva's structural approach is considered "standard," it was primarily cited in only one of the six communities they identified.

Overall, while there are some common elements, sociologists continue to debate and refine how to best conceptualize and study racism in contemporary society. The field appears to be moving toward more multidimensional understandings that can capture both overt and subtle manifestations of racism at individual, institutional, and cultural levels.

## Examining research on racism

We conceptualize a comparative systematic review (CSR) method to identify variations in bibliometric data on a topic of study. We focus comparisons on three sample data sets which contains articles on racism in the context of science, technology, engineering, and mathematics. Rather than generate increasingly defined criteria to inform the analysis, we focus on the study index within the bibliographic metadata. We find significant differences in the content and associations in author and keyword data, as well as occurrence networks and study themes. Those themes which occur across data sets are framed as notions, and posiiton as a part of a conceptual replication.

CSR responds to the completeness dilemma in the analysis of bibliographic metadata. As research productivity increases and different lines of study are leveraged across different disciplines, CSR supports researchers in their development of inclusion and exclusion criteria for systematic reviews on increasingly mulitdisiplineary, interdisciplinary, and cross-disciplinary topics of study. We describe the differences of these frameworks as one example to situating the objects of data for a systematic analysis.

We use the term 'notion' to examine the driving themes for the study. We examine context-dependent instances of racism that point to broader meanings across two sets of disciplinary outputs, in the education research indices and in science, technology, engineering, and mathematics (STEM) publication indices. We make sense of noticeable differences in the way that scholars have utilized racism across the disciplines and different scholarly outlets.

Beverly Daniel Tatum describes racism as "a system of advantage based on race" and adds that it is "not only a personal ideology based on racial prejudice, but a system involving cultural messages and institutional policies and practices as well as the beliefs and actions of individuals" (Tatum, 1997, p. 7).

Ruth Wilson Gilmore defines racism as "the state-sanctioned or extralegal production and exploitation of group-differentiated vulnerability to premature death" (Gilmore, 2007, p. 28).

Ibram X. Kendi defines racism as "a marriage of racist policies and racist ideas that produces and normalizes racial inequities" (Kendi, 2019, p. 18).

> Racism is a structure, a network of relations at social, political, economic, and ideological levels that shapes the life chances of the various races (Bonilla-Silva, 2013, p. 26)

# Conceptual Framework

This study is motivated, in part, by the work of Shiao & Woody (2021) who discuss the various meanings of racism in sociological journal content.
For the purpose of this study, our team sought to make sense of the nature of this intellectual structure in STEM, and the various concepts and themes that are situated within this structure.


Table: Conceptualizations and meanings of racism

| **Bonilla-Silva** | **Shiao and Woody**                 | 
|:-------|:------------|
| A: *Pyschological*   | A: Individual attitudes                |
| B: *Cultural*        | B: Cultural schema                     |
| C: *Structural*      | C$_1$: Structural: Pre-existing conditions |
|                   | C$_2$: Structural: Create or maintain      |
|                   | C$_3$: Structural: Spatial                 |

## Replication

### Bibliographic collection

*Data source*: Clariate Analytics Web of Science

*Data format*: Plain text

*Time span*: 2003 - 2023

*Document Type*: Articles; Articles, early access

*Query date*: July 1, 2024

## Research question

*What is the state of the intellectual structure and emerging notions in research on racism in STEM?*

# Methods and Data

## Methodological Framework

We integrate systematic review techniques with computational methods to generate the methods for the study. The systematic review process builds on qualitative meta-synthesis techniques, as well as content and thematic analysis. The computational methods follow the general scientific mapping workflow for working with large sets of citations data.

### Database assessment

Four databases were used initially to identify the possible collections of sources, and to compare the various results obtained across databases: ERIC, Google Scholar, Scopus, and Web of Science.

**EBSCO** (Five databases total: Academic  Search  Premier,  Social Sciences  Full  Text  (H. W. Wilson), Education Research Complete, ERIC, and APA PsychArticles and APA PsychInfo Collection).

**ERIC**.

**Scopus**.

**Google Scholar**.

We noticed across these other databases which provide APIs to support the easy access to bibliometric data that variation would be a concern. As a result, we proceeded further with the conceptual replication by using the Web of Science Core Collection of bibliometric data. This selection provided us with an opportunity to examine one of the main goals of the study, which was to understand differences across disciplinary boundaries despite the increasingly interdisciplinary nature of the research on racism in STEM.


Table: Results of of database searches^[1]

Pattern | EBSCO | ERIC | Scopus | Google Scholar | Web of Science
:----|:----|:----|:----|:----|:----
XXXX|XXXX|XXXX|XXXX|XXXX|XXXX
XXXX|XXXX|XXXX|XXXX|XXXX|XXXX
XXXX|XXXX|XXXX|XXXX|XXXX|XXXX
XXXX|XXXX|XXXX|XXXX|XXXX|XXXX
XXXX|XXXX|XXXX|XXXX|XXXX|XXXX
XXXX|XXXX|XXXX|XXXX|XXXX|XXXX

[1 All final searches took place on July 15, 2024]

**Web of Science (WoS)**. The WoS Core Collection was used to obtain bibliographic information for the study sample. The Science Citation Index Expanded (SCI-EXPANDED), Social Science Citation Index (SSCI), Arts & Humanities Citation Index (ACHI), and Emerging Sources Citation Index (ESCI) were selected. The Conference Proceedings Citation Index - Science (CPCI-S), Conference Proceedings Citation Index - Social Science (CPCI-SSH), Book Citaton Index - Science (BKCI-S), and Book Citation Index - Social Science & Humanities (BKCI-SSH) were not selected. Given the formal inclusion of psychology as one of the STEM disciplines by research foundations, such as the National Science Foundation (NSF) in the United States, social science bibliographic entries were included in the second sample.

[Query link]{<https://www.webofscience.com/wos/woscc/summary/6d0d4c0d-2e83-4162-bd8f-077ad0a4cc34-f7d6e8a1/relevance/1>}

### Bibliometric analysis.

#### Pilot analysis: Year span: 1995-2015

We conducted a replication of the Shian & Woody study. Study details were obtained from the original article, as well as cited or related articles, to ensure a set of  steps for the conceptual replication were derived. The initial search for articles from the Web of Science Core Collection included the query (((TI=(racism or racist)) OR AB=(racism or racist)) OR AK=(racism or racist)) AND DOP=(1995-01-01/2015-12-31)) AND LA=(English) returned 8,950 results. When the article requirement and individual collection indices were added (SSCI, A&HCI, ESCI, and SCI-EXPANDED), a total of 6,378 articles remained. When the results were subset to those indexed as Sociology, a total of 1,033 articles remained.

(TI=(racism or racist) OR AB=(racism or racist) OR AK=(racism or racist)) AND DOP=(1995-01-01/2015-12-31) AND LA=(English)

The search results returned a total of 1,430 articles, four less than the original study. The results were not able to duplicated was likely an issue of database updates. These changes were within an appropriate range to proceed. 

When education journal indexed articles were selected, a total of 853 results remained.
<https://www.webofscience.com/wos/woscc/summary/bfe60ccd-86c6-40d0-9ba5-16b667ff199e-f83fbe53/relevance/1>

When science, technology, engineering, and mathematical indexed subjects were selected a total of 1,462 results remained.
<https://www.webofscience.com/wos/woscc/summary/eff6a28e-2893-454e-a79f-13e94a7d6c0e-f83fba7d/relevance/1>

Forty results were returned when the sample is subset to education journal articles including one of the terms science, technology, engineering, mathematics, or STEM in their title or abstract.
<https://www.webofscience.com/wos/woscc/summary/d5c643f1-25c7-46c1-93ef-6f458652d262-f83fe3c7/relevance/1>

### Qualitative metasyntehsis

#### Inclusion and exclusion criteria

We use an integrated appraisal method that takes into account the standards of bibliometric analysis with the goals of qualitative metasynthesis (Berry & Thunder, 2013).

#### Integrated appraisal method and selection


### Analytic sample

A final total analytic sample of 351 articles was selected for inclusion in the study.

These articles included the words racism or racist in their abstracts and at least one of the terms, science, technology, engineering, mathematics, or STEM in their abstracts); the time period was expanded to include article for the 20 year period from 2003 to 2023.
<https://www.webofscience.com/wos/woscc/summary/331dedce-0c5c-40cd-b63e-734077ea3736-f840d6cf/relevance/1>

This expansion was based on the social contexts of major research outputs across science and mathematics education. Namely, the social turn in mathematics education was published in 2000. No Child Left Behind Act was enacted in 2001.
There was also a spike in research productivity identified in the study samples during the cleaning and initial analysis (which included all years) for the specific period between 2000 and 2003.

Final data analysis was run on July 3, 2024.

Prior to this period, multiple programs were designed and checked to ensure accuracy across different samples and generated psuedo data.
All programs were confirmed to return the same results across studies.

3,318,115 results from Science Citation Index Expanded (SCI-EXPANDED), Social Sciences Citation Index (SSCI), Arts & Humanities Citation Index (A&HCI), Emerging Sources Citation Index (ESCI): for ABSTRACT AND racism or science or technology or engineering or mathematics or stem <https://www.webofscience.com/wos/woscc/summary/b7144fe8-2337-4d15-b2db-0fc78da08022-f78ebcfe/relevance/1>
